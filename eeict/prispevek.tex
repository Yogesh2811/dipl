%--------------------------------------------------------------------------------
%
% Text p��sp�vku do sborn�ku EEICT
%
% Vytvo�il:  Martin Drahansk�
% Datum:     26.02.2007
% E-mail:    drahan@fit.vutbr.cz
%
%--------------------------------------------------------------------------------
%
% P�elo�en�: pdflatex prispevek.tex
%
% Optim�ln� zp�sob pou�it� = p�epi�te jen vlastn� text
%
\documentclass{eeict}
\inputencoding{cp1250}
\usepackage[bf]{caption2}
%--------------------------------------------------------------------------------

\title{TITLE OF THE PAPER - IN ENGLISH}
\author{Jan Wozniak}
\programme{Master Degree Programme (2), FIT BUT}
\emails{xwozni00@stud.fit.vutbr.cz}

\supervisor{Peter Jurne�ka}
\emailv{ijurnecka@fit.vutbr.cz}

\abstract{Encryption of real-time multimedia data transferes is one
  of the tasks for telecommunication infrastructure which should be considered
  in order to provide essential level of security. Execution time of ciphering 
  algorithm could play fundamental role in delay of the packets, therefore, it 
  provides interesting challenge in terms of optimization methods. This work 
  focuses on paralellization possibilities of processing SRTP for the 
  purposes of private gateway with the usage of OpenCL framework, utilization
  gateway`s resources and analysis of potential improvement.}
\keywords{AES, SRTP, general-purpose GPU, OpenCL, parallel computations, 
  gateway, VoIP.}

\begin{document}
% -- Hlavi�ka pr�ce --

\maketitle

%-------------------------------------------------------------------------------
\selectlanguage{english}
\section{Introduction}
One of the essential metrics for measuring VoIP Gateway performance is 
the number of simultaneous calls. It is affected mostly by the
computational demands of used communication protocols and number of registered
users. While the count of registered users provides very limited room for 
improvement by the nature of the problem itself, there could be wide variety
of aproaches in implementing the protocol stacks. 

Significant amount of resources are utilized during indirect simultaneous call
sessions by processing multimedia packets. Since security has recently grown 
to be necessary feature in VoIP communication, and the encryption and decryption
processes are designed with the idea of optimization, it is primary scope of 
interest of this paper.

%Development and results in the areas of parallel architectures shows that many
%procedures could be distinctively accelerated by executing the algorithm on the
%processing unit capable of parallel computations. Therefore, this paper aims on
%implementation and analysis of parallel processing of encrypted real-time 
%multimedia data transferes.

\section{SRTP Processing}
Secure Real-time Transport Protocol was designed as an extension over RTP protocol
to obtain security and confidentiality for multimedia sessions. 

-- packet consists of multiple independent blocks.

\subsection{AES}
-- parallelization of computation of cells in AES block.

\subsection{Persistent Thread}
-- larger kernels to minimize negative influence of startup execution of the kernel.

\section{Gateway Design}
-- integration of SRTP stack to the VoIP gateway.

\section{Conclusion}
-- results of testing.



%------------
% Citace
%
\begin{thebibliography}{9}
%  \bibitem{rybicka} Rybi�ka, J.: \LaTeX pro za��te�n�ky, Brno, Konvoj 1999,
%            ISBN 80-85615-77-0
%  \bibitem{orsag} Ors�g, F.: Vision f�r die Zukunft. Biometrie, Kreutztal,
%            DE, b-Quadrat, 2004, s. 131-145, ISBN 3-933609-02-X
%  \bibitem{drahansky} Drahansk�, M., Ors�g, F.: Biometric Security Systems:
%    Robustness of the Finger-print and Speech Technologies. In: BT 2004 - 
%    International Workshop on Biometric Technologies, Calgary, CA, 2004,
%    s. 99-103
\end{thebibliography}

\end{document}
