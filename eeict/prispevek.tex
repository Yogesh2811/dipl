%--------------------------------------------------------------------------------
%
% Text p��sp�vku do sborn�ku EEICT
%
% Vytvo�il:  Martin Drahansk�
% Datum:     26.02.2007
% E-mail:    drahan@fit.vutbr.cz
%
%--------------------------------------------------------------------------------
%
% P�elo�en�: pdflatex prispevek.tex
%
% Optim�ln� zp�sob pou�it� = p�epi�te jen vlastn� text
%
\documentclass{eeict}
\inputencoding{cp1250}
\usepackage[bf]{caption2}
\usepackage{url}

%--------------------------------------------------------------------------------

\title{Design and Performance Analysis of Parallel Processing of SRTP Packets}
\author{Jan Wozniak}
\programme{Master Degree Programme (2), FIT BUT}
\emails{xwozni00@stud.fit.vutbr.cz}

\supervisor{Peter Jurne�ka}
\emailv{ijurnecka@fit.vutbr.cz}

\abstract{Encryption of real-time multimedia data transferes is one
  of the tasks for telecommunication infrastructure which should be considered
  in order to reach essential level of security. Execution time of ciphering 
  algorithm could play fundamental role in delay of the packets, therefore, it 
  provides interesting challenge in terms of optimization methods. This work 
  focuses on paralellization possibilities of processing SRTP for the 
  purposes of private gateway with the usage of OpenCL framework, utilization
  gateway's resources and analysis of potential improvement.}
\keywords{AES, SRTP, general-purpose GPU, OpenCL, parallel computations, 
  gateway, VoIP.}

\begin{document}
% -- Hlavi�ka pr�ce --

\maketitle

%-------------------------------------------------------------------------------
\selectlanguage{english}
\section{Introduction}
One of the essential metrics for measuring VoIP gateway's performance is 
the number of simultaneous calls. It is affected mostly by the
computational demands of used communication protocols and number of registered
users. While the count of registered users provides very limited room for 
improvement by the nature of the problem itself, there could be wide variety
of aproaches in implementing the protocol stacks. 

Significant amount of resources are utilized during indirect simultaneous call
sessions by processing multimedia packets. Since security has recently grown 
to be necessary feature in VoIP communication, and the encryption and decryption
processes are designed with the idea of optimization, it is primary scope of 
interest of this paper.

%Development and results in the areas of parallel architectures shows that many
%procedures could be distinctively accelerated by executing the algorithm on the
%processing unit capable of parallel computations. Therefore, this paper aims on
%implementation and analysis of parallel processing of encrypted real-time 
%multimedia data transferes.

\section{SRTP Processing}
Secure Real-time Transport Protocol was designed as an extension over RTP protocol
to obtain security and confidentiality for multimedia sessions on application layer
of ISO/OSI model. The packet has usual structure cosisting of the SRTP header,
authentication extension and only encrypted data payload. The default, and as
this paper was written, the only defined cipher is 128bit AES. 

In VoIP communication the time has essential impact on the quality of
transmitted information, therefore, it is important that ensuring the 
security of RTP wouldn't increase the latency over the acceptable level.
Among typical limitations of real-time communications 
belong \cite{perkins:rtp2003}:
\vspace{-0.5em}
\begin{itemize}
\vspace{-0.5em}
\item Maximal tolerable latency of round-trip time 300ms.
\vspace{-0.5em}
\item Smaller packet loss than 5\%.
\vspace{-0.5em}
\item Sensitivity to factors that are difficult to objectively measure
such as jitter.
\end{itemize}
\vspace{-0.5em}

\begin{figure}[h!]
\centering
\includegraphics[width=14cm]{eeit.pdf}
\caption{SRTP Processing Scheme: the data from network are captured in the \textit{network interface}
running in its own thread. With corresponding \textit{stream}, input and output buffer
are passed to the \textit{daemon} which can select one of three types of \textit{SRTP parser} 
according to the application configuration. Index to output buffer is passed back to the
\textit{network interface} and result is sent to the desired address.}
\label{lcpstack}
\end{figure}

The exact size of payload in SRTP packet can differ widely according to the 
used codec, its bit rate, and sampling frequency. The basic multimedia codec 
is G.711, which should be supported by every multimedia device and with standardly 
used 20ms sampling period, the length of payload is 160 bytes. Due to it's wide
support it has been chosen and used for evaluation and comparison of two aproaches 
for encryption and decryption.

\subsection{AES}
The counter-mode for AES has been selected for the VoIP sessions due to its invariacy
to delay and even possible loss of packets. This provides flexibility in SRTP processing
for packets received out of order as well. 

Each byte in the block of AES cipher can be computed separately. The local dependency
of bytes is limited to the distinct steps of the algorithm which requires usage of
barriers for local synchronization.

\subsection{Persistent Thread}
For massive parallel applications the obvious aproach would be to utilize as much
of machine's power as possible to gain the largest speed-up in every single
execution. However, the aim of this paper is to minimize large delays for multiple 
sessions which requires rather carefull allocation of resources. Persistent threads
is special type of programing paradigm combining both, the possible gain of mapping
the program for parallel computation and considerate usage of resources \cite{pt}.

Since the initialization of computational kernel can consume significant amount of
time compared to the actual execution, larger kernel reusing its resources
for multiple similar computations could render the initialization negligible trading
off portion of parallelization.

This approach has been chosen for packet parsing, while instead of mapping 160 OpenCL
work-items on the entire G.711 packet's payload it uses 16 work-items in a loop that 
goes through the entire packet.

\section{Conclusion}
The commercial gateway with optimized hardware can hold around 120 concurrent
calls\footnote{\url{http://www.athlsolutions.com/web/en/Products/tabid/128/ProdID/38/Hipath\_4000.aspx}}. 
The implementation proposed as backup for this paper evaluation can be sumarized
in following graphs of distributed packet latencies. Measured was round-trip time latency
of each packet during 50 to 150 concurrent calls that all lasted 20 seconds. 
The range of latencies seem to be unreliable however more than 95\% of the
recceived packets falls between the smaller range of the column which is
visualized thicker.

The tests were all done on the machine with processor intel i5 2500k with HD3000 graphics chip
running OpenSUSE 12.2 and OpenCL version 1.2.

\begin{figure}[h!]
\centering
\includegraphics[width=13cm]{gpu.png}

\vspace{4em}

\includegraphics[width=13cm]{cpu.png}
\caption{Top graph visualizes delay for prallel processing using OpenCL, bottom graph
is for serial processing.}
\label{lcpstack}
\end{figure}

%------------
% Citace
%
%\begin{thebibliography}{9}
%  \bibitem{rybicka} Rybi�ka, J.: \LaTeX pro za��te�n�ky, Brno, Konvoj 1999,
%            ISBN 80-85615-77-0
%  \bibitem{orsag} Ors�g, F.: Vision f�r die Zukunft. Biometrie, Kreutztal,
%            DE, b-Quadrat, 2004, s. 131-145, ISBN 3-933609-02-X
%  \bibitem{drahansky} Drahansk�, M., Ors�g, F.: Biometric Security Systems:
%    Robustness of the Finger-print and Speech Technologies. In: BT 2004 - 
%    International Workshop on Biometric Technologies, Calgary, CA, 2004,
%    s. 99-103
%\end{thebibliography}

\bibliography{literatura} % viz. literatura.bib
\bibliographystyle{plain}
\end{document}
